\section{Espionagem e Software Livre}

\begin{frame}\frametitle{Introdução}

A espionagem através de softwares e internet é o assunto do momento, que
estorou quando Edward Snowden vazou documentos que mostravam que o
governo americano tinha acesso aos dados de vários países inclusive o
Brasil.

Mas será se isso é de hoje ou faz tempo que somos vigiados e só agora
sabemos?

E ainda, a que ponto sermos espionados nos causa algum prejuízo?

\end{frame}

\begin{frame}[fragile]\frametitle{Acesso seguro através da internet}

\url{http://tools.ietf.org/html/rfc6101}

Para acessar a internet com segurança usamos o protocolo SSL junto com o
HTTP.

\url{http://webcache.googleusercontent.com/search?q=cache:ZTuIJfCKCzwJ:www.cse.buffalo.edu/DBGROUP/nachi/ecopres/fengmei.ppt+&cd=4&hl=pt-BR&ct=clnk&gl=br&client=ubuntu}

Entretanto isso é uma especificação, ou seja, alguém tem que
implementar. Algumas empresas que o implementam:

\begin{itemize}
\item
  OpenSSL (http://www.openssl.org/)-- Provides Information about a free,
  open-source implementation of SSL.
\item
  Apache-SSL (http://www.apache-ssl.org/)-- Describes Apache-SSL, a
  secure Webserver, based on Apache and SSLesy/OpenSSL.
\item
  SSLeay (ftp://ftp.uni-mainz.de/pub/internet/security/ssl/SSL/) -- a
  free implementation of Netscape's Secure Socket Layer
\item
  Planet SSL
  (http://www.rsasecurity.com/standards/ssl/developers.html)-- provides
  C-programs and Java-programs of SSL.
\end{itemize}
Exemplo de mesma implementação para problemas diferentes:

multiplicação de 2 números:

2 * 3

ou

2 + 2 + 2

Daí o termo HTTPS

O google tem sua própria autoridade certificadora.

A Validade desse tipo de protocolo é de 1 ou 3 anos. Por quê?

\url{http://www.inet2000.com/public/encryption.htm}
\url{http://www.digicert.com/TimeTravel/math.htm}

\end{frame}

\begin{frame}[fragile]\frametitle{Hackers e Crakers}

\url{http://www.pctools.com/security-news/crackers-and-hackers/}

\url{http://arstechnica.com/security/2013/10/how-the-bible-and-youtube-are-fueling-the-next-frontier-of-password-cracking/1/}

\begin{block}{LinkedIn attack}

\url{http://www.ma.rhul.ac.uk/static/techrep/2013/MA-2013-07.pdf}

\end{block}

\begin{block}{backdoors}

\url{http://www.devttys0.com/2013/10/reverse-engineering-a-d-link-backdoor/}
\url{http://seclists.org/fulldisclosure/2013/Nov/76}

\url{http://www.revista.espiritolivre.org/canyouseeme-org-utilitario-para-checar-portas-abertas}

\end{block}

\end{frame}

\begin{frame}[fragile]\frametitle{Hacking google and duck duck go!}

\url{http://netsecurity.about.com/cs/generalsecurity/a/aa070303.htm}

\begin{block}{google: buscas específicas}

\begin{itemize}
\item
  comsolid filetype:pdf
\item
  inkscape site:comsolid.org
\end{itemize}
\end{block}

\begin{block}{ddg goodies}

\begin{itemize}
\item
  !google
\item
  expand http://va.mu/dIwg
\item
  hello world python
\item
  ip address
\item
  github latexila
\item
  @comsolid
\item
  61900
\item
  age of linus torvalds
\item
  define free software
\item
  reverse dilosmoc
\end{itemize}
\end{block}

\begin{block}{http://duckduckhack.com/}

\end{block}

\end{frame}

\begin{frame}[fragile]\frametitle{Ataque ao kernel}

\url{http://linux.slashdot.org/story/13/10/09/1551240/the-linux-backdoor-attempt-of-2003}

\url{https://freedom-to-tinker.com/blog/felten/software-transparency/}

\end{frame}
